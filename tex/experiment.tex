% !TeX spellcheck = ru_RU
\chapter{Исследовательский раздел}

В данном разделе будет проведено нагрузочное тестирование разработанного ПО и сервера nginx при помощи Apache Benchmark и сравнение результатов.

\section{Технические характеристики}

Технические характеристики устройства, на котором выполнялись измерения:

\begin{enumerate}[label=\arabic*.]
	\item операционная система Ubuntu, 20.04.4 \cite{ubuntu};
	\item память 8 ГБ;
	\item процессор 2,4 ГГц 4‑ядерный процессор Intel Core i5-1135G7 \cite{intel}.
\end{enumerate}

Во время замеров ноутбук был включен в сеть электропитания, нагружен только встроенными приложениями окружения и разработанным сервером.

\section{Нагрузочное тестирование}

Нагрузочное тестирование проводилось при помощи утилиты apache benchmark. На главную страницу сервера посылалось единовременно различное количество запросов от 10 до 1000. Рассматривалось среднее время обработки одного запроса.

Команда для единовременной отправки 100 запросов на разработанный сервер: \newline \textbf{ab -n 10000 -c 100 http://localhost:8080}.

Команда для единовременной отправки 100 запросов на сервер nginx: \newline \textbf{ab -n 10000 -c 100 http://localhost:8080}.

На рисунке \ref{img:graphic} представлены результаты измерений.

\pagebreak

%\img{1}{graphic}{Результаты тестирования}

По результатам тестирования разработанный сервер уступает nginx на 20\% при одновременном обращении до 500 клиентов. С увеличением количества одновременных обращений среднее время выдачи ответа у разработанного сервера растёт медленнее, чем у сервера nginx и к 1000 клиентам разница уже составляет около 5\% или 1 микросекунду.

\section*{Вывод}

В данном разделе было проведено нагрузочное тестирование разработанного ПО и сервера nginx при помощи Apache Benchmark и сравнение результатов.
По результатам тестирования разработанный сервер немного уступает в среднем времени обработки запроса серверу nginx.