% !TeX spellcheck = ru_RU
\chapter{Технологический раздел}

В данном разделе будут приведены средства реализации программного обеспечения, представлены листинги исходного кода.

\section{Средства реализации}

Для реализации ПО был выбран язык C, согласно требованиям к выполнению курсовой работы.

\section{Реализация алгоритмов}
В листингах \ref{listing_server}-\ref{listing_server_3} представлена реализация основного алгоритма работы сервера. 
А в листингах \ref{listing_http}-\ref{listing_http_2} представлена реализация обработки запросов на сервер.

	\begin{lstlisting}[caption=Основной алгоритм работы сервера]
int init_server() {
	close(STDIN_FILENO);
	close(STDERR_FILENO);
	signal(SIGCHLD, sigchld_handler);
	worker_count = get_nprocs();
	
	servsock = socket(AF_INET, SOCK_STREAM, 0);
	if (servsock == -1) {
		log_error("Cannot create socket");
		return EXIT_FAILURE;
	}
	if (setsockopt(servsock, SOL_SOCKET, SO_REUSEADDR, &(int){ 1 }, sizeof(int)) == -1) {
		close(servsock);
	\end{lstlisting}
	
	\begin{lstlisting}[caption=Основной алгоритм работы сервера (продолжение)]
		log_error("Cannot set socket to reuse address");
		return EXIT_FAILURE;
	}
	struct sockaddr_in servaddr;
	servaddr.sin_family = AF_INET;
	servaddr.sin_addr.s_addr = inet_addr(HOST);
	servaddr.sin_port = htons(PORT);
	if (bind(servsock, (struct sockaddr *)&servaddr, sizeof(servaddr)) == -1) {
		close(servsock);
		log_error("Cannot bind address");
		return EXIT_FAILURE;
	}
	
	max_fd = servsock;
	rfds = malloc(sizeof(fd_set));
	FD_ZERO(rfds);
	FD_SET(servsock, rfds);
	workers = malloc(sizeof(worker_t) * worker_count);
	for (size_t i = 0; i < worker_count; i++) {
		workers[i].servsock = servsock;
		if (fork_worker(workers + i) == EXIT_FAILURE) {
			free(workers);
			return EXIT_FAILURE;
		}
	}
	return EXIT_SUCCESS;
}
	\end{lstlisting}

\pagebreak

	\begin{lstlisting}[caption=Обработка запроса сервером]
void handle_connection(connection_t *connection) {
	connection->is_served = 1;
	int fd = connection->fd;
	memset(connection->request.buffer, 0, REQUEST_SIZE_LIMIT + 1);
	read(fd, connection->request.buffer, REQUEST_SIZE_LIMIT - 1);
	
	
	sscanf(
	connection->request.buffer, "%s %s", connection->request.method, connection->request.path);
	
	if (strcmp(connection->request.method, "GET") == 0) {
		log_info("GET request");
		struct stat sb;
		char actual_path[PATH_SIZE_LIMIT + 3];
		if (strcmp(connection->request.path, "/") == 0) {
			sprintf(actual_path, "./index.html");
		} else {
			sprintf(actual_path, ".%s", connection->request.path);
		}
		
		int rc;
		if ((rc = stat(actual_path, &sb)) == 0) {
			const char *mime_type = detect_mime_type(actual_path);
			sprintf(connection->response.status, RESPONSE_TEMPLATE, 200, "OK");
			sprintf(connection->response.headers, CONTENT_TYPE_TEMPLATE, mime_type, sb.st_size);
		} else {
			sprintf(connection->response.status, RESPONSE_TEMPLATE, 404, "Not Found");
			write(fd, connection->response.status, strlen(connection->response.status));
		}
		if (rc == 0) {
			FILE *file = fopen(actual_path, "rb");
			if (file == NULL && errno == EACCES) {
				sprintf(connection->response.status, RESPONSE_TEMPLATE, 403, "Forbidden");
			}
		\end{lstlisting}
		
		
		\begin{lstlisting}[caption=Обработка запроса сервером (продолжение)]
			write(fd, connection->response.status, strlen(connection->response.status));
			write(fd, connection->response.headers, strlen(connection->response.headers));
			write_file(file, fd);
			fclose(file);
		}
		
	} else if (strcmp(connection->request.method, "HEAD") == 0) {
		log_info("HEAD request");
		struct stat sb;
		char actual_path[PATH_SIZE_LIMIT + 3];
		sprintf(actual_path, ".%s", connection->request.path);
		if (stat(actual_path, &sb) == 0) {
			const char *mime_type = detect_mime_type(actual_path);
			sprintf(connection->response.status, RESPONSE_TEMPLATE, 200, "OK");
			sprintf(connection->response.headers, CONTENT_TYPE_TEMPLATE, mime_type, sb.st_size);
		} else {
			sprintf(connection->response.status, RESPONSE_TEMPLATE, 404, "Not Found");
		}
		
		write(fd, connection->response.status, strlen(connection->response.status));
		write(fd, connection->response.headers, strlen(connection->response.headers));
	} else {
		log_info("Unknown request");
		sprintf(connection->response.status, RESPONSE_TEMPLATE, 405, "Method Not Allowed");
		write(fd, connection->response.status, strlen(connection->response.status));
	}
}
\end{lstlisting}

\section{Пример работы программы}

На рисунке \ref{work} представлен пример работы программы, а именно основная страница, отображаемая при обращении к серверу.

\pagebreak

%\img{0.3}{work}{Пример работы программы}

\section*{Вывод}

В этом разделе был выбран язык программирования, представлены листинги исходного кода.