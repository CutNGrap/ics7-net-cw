% !TeX spellcheck = ru_RU
\chapter{Аналитический раздел}

В данном разделе проводится анализ предметной области, анализ шаблона управления потоками threadpool, модели асинхронного блокирующего ввода-вывода, системного вызова select, а также протокола передачи данных http. Формулируются требования к разрабатываемому приложению.

\section{Статический сервер}

Статический сервер --- программа, принимающая запросы по протоколу http и возвращающая на них ответы со статической информацией.

Статическая информация --- информация, которая вообще или редко подвергается изменениям. В данной работе предметом раздачи сервера будут файлы разных форматов:
\begin{enumerate}
	\item HTML (от англ. HyperText Markup Language) --- гипертекстовые документы;
	\item CSS (от англ. Cascading Style Sheets) --- файлы стилей;
	\item JS (от англ. JavaScript) --- файлы с кодом на языке java script;
	\item PNG (от англ. Portable Network Graphics) --- файлы растровых изображений;
	\item JPEG (от англ. Join Photographic Experts Group) --- файлы растровых изображений;
	\item SWF (от англ. Small Web Format) --- файлы векторной графики;
	\item GIF (от англ. Graphics Interchange Format) --- файлы растровых изображений;
	\item TXT (от англ. text) --- файлы с текстом.
\end{enumerate}

\section{Готовые решения}

Так как любой веб-сайт нуждается в сервере, а с момента запуска сети Интернет прошло уже более 30 лет, то на данный момент существует множество вариантов статических серверов.

NGINX (от англ. Engine X)~\cite{nginx} --- это HTTP-сервер и обратный прокси-сервер, почтовый прокси-сервер, а также TCP/UDP прокси-сервер общего назначения, написанный Игорем Сысоевым, выпускником МГТУ им. Н.Э.Баумана. Согласно статистике Netcraft~\cite{netcraft} nginx обслуживал или проксировал 20.72\% самых нагруженных сайтов в декабре 2023 года.

Apache (от англ. a patchy server)~\cite{apache} --- веб-сервер с открытым исходным кодом. Является одним из первых решений в данной области и до появления nginx обслуживал до 70\% всех приложений, отслеживаемых Netcraft~\cite{netcraft}. Основным достоинством является гибкость конфигурации, позволяющая подключать внешние модули для предоставления данных, модифицировать данные об ошибках и т. д. На данный момент обслуживает около 22\% веб-серверов.

\section{Шаблон prefork}

Шаблон управления потоками prefork (предварительный запуск процессов) является альтернативой пулу потоков и используется в том числе в веб-серверах.

В отличие от пула потоков, где создается набор потоков при запуске программы, в шаблоне pefork процессы создаются заранее. Каждый процесс может обрабатывать входящие запросы, и родительский процеcс управляет количеством процессов в пуле. 

Этот подход подходит для серверов, которым необходимо избегать потоков из-за несовместимости с непотоково-безопасными библиотеками.

Шаблон prefork на статическом сервере работает путем создания заранее определенного количества процессов, которые ожидают и обслуживают входящие запросы на статические ресурсы, такие как HTML, CSS и изображения. Когда запрос поступает, один из доступных процессов берет на себя обработку запроса, что позволяет обрабатывать несколько запросов параллельно. Это позволяет увеличить производительность сервера за счет параллельной обработки запросов.

\section{Мультиплексирование}

Мультиплексирование --- модель асинхронного блокирующего ввода-вывода. Основывается на опросе набора источников о готовности.
Модель является блокирующей, так как главный процесс блокируется в ожидании готовности одного из источников.
Асинхронность достигается за счёт того, что главный поток производит одновременный опрос сразу нескольких источников, и блокируется только до готовности одного из них.

\subsection{Системный вызов pselect}

Системный вызов \textit{pselect} --- функция, существующая в Unix-подобных и POSIX системах и предназначенная для опроса файловых дескрипторов открытых каналов ввода-вывода. Подключение данной функции происходит при помощи заголовочного файла sys/select.h на языке программирования Си.

Функция pselect принимает на вход 6 аргументов: 

\begin{enumerate}[label=\arabic*)]
	\item nfds, число типа int, значение котого вычисляется как $n + 1$, где $n$ --- максимальное значение файлового дескриптора из наборов файловых дескрипторов, опрос которых осуществляется;
	\item readfds, набор файловых дескрипторов типа fd\_set, предназначенных для опроса на предмет чтения из них;
	\item writefds, набор файловых дескрипторов типа fd\_set, предназначенных для опроса на предмет записи в них;
	\item exceptfds, набор файловых дескрипторов типа fd\_set, предназначенных для опроса на предмет появления исключительных ситуаций;
	\item timeout, структура типа struct timespec (содержит миллисекунды и наносекунды), определяет максимальный интервал времени, в течение которого будет осуществлено ожидание;
	\item sigmask, маска типа sigset\_t сигналов, позволяющая изменить маску сигналов на время работы функции select.
\end{enumerate}

Возвращает функция количество файловых дескрипторов, готовых к вводу-выводу, 0 в случае, если за интервал времени timeout таких дескрипторов нет и -1 в случае ошибки.

При использовании в разрабатываемой программе необходимыми будут лишь параметры nfds, readfds и timeout, поскольку при раздаче статической информации требуется чтение с диска в определённый промежуток времени, а не запись на диск или мониторинг исключительных ситуаций на сокетах.


\section{HTTP}

HTTP (от англ. HyperText Transfer Protocol) --- протокол уровня приложений сетевой модели OSI~\cite{OSI}, предложенной международной организацией по стандартизации ISO~\cite{ISO}.
Это текстовый протокол, изначально предназначенный для передачи гипертекстовых документов.

Каждое HTTP-сообщение состоит из трёх частей: стартовая строка, заголовки, тело сообщения.
В стартовой строке указывается один из методов запроса: OPTIONS, GET, HEAD, PUT, POST, PATCH, DELETE, TRACE, CONNECT.
В данной работе будут рассмотрены только два из них, GET и HEAD.

Метод GET используется для запроса содержимого ресурса. Запросы этого метода считаются идемпотентными, то есть на один и тот же запрос всегда выдаётся один и тот же ответ.

Метод HEAD аналогичен запросу GET, за исключением того, что в ответе на этот запрос отсутствует тело.

В заголовках HTTP-сообщения указываются определённые свойства и характеристики как тела сообщения, так и установленного соединения.
Наример, один из заголовков, content-type, устанавливает тип данных, передаваемых в теле.
Для каждого формата файла он свой:
\begin{enumerate}
	\item HTML --- text/html;
	\item CSS --- text/css;
	\item JS --- text/javascript;
	\item PNG --- image/png;
	\item JPEG --- image/jpeg;
	\item SWF --- application/x-shockwave-flash;
	\item GIF --- image/gif;
	\item TXT --- text/plain.
\end{enumerate}

\section{Требования к разрабатываемой программе}

На основе задания в курсовой работе и вышеперечисленного разрабатываемая программа должна соответствовать следующим требованиям:
\begin{enumerate}[label=\arabic*.]
	\item поддержка запросов GET и HEAD (поддержка статусов 200, 403, 404);
	\item ответ на неподдерживаемые запросы статусом 405;
	\item выставление content type в зависимости от типа файла (поддержка .html, .css, .js, .png, .jpg, .jpeg, .swf, .gif);
	\item корректная передача файлов размером в 100мб;
	\item сервер по умолчанию должен возвращать html-страницу на выбранную тему с css-стилем;
	\item учесть минимальные требования к безопасности статик-серверов (предусмотреть ошибку в случае если адрес будет выходить за root директорию сервера);
	\item реализовать логгер;
	\item использовать язык Си. Сторонние библиотеки запрещены;
	\item реализовать архитектуру с использованием prefork и pselect;
	\item статик сервер должен работать стабильно.
\end{enumerate}

\section*{Вывод}

В данном разделе был проведён анализ предметной области, анализ шаблона управления потоками prefork, модели асинхронного блокирующего ввода-вывода, системного вызова pselect, а также протокола передачи данных http. Сформулированы требования к разрабатываемому приложению.


